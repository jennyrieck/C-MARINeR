\documentclass[man]{apa6}
\usepackage{lmodern}
\usepackage{amssymb,amsmath}
\usepackage{ifxetex,ifluatex}
\usepackage{fixltx2e} % provides \textsubscript
\ifnum 0\ifxetex 1\fi\ifluatex 1\fi=0 % if pdftex
  \usepackage[T1]{fontenc}
  \usepackage[utf8]{inputenc}
\else % if luatex or xelatex
  \ifxetex
    \usepackage{mathspec}
  \else
    \usepackage{fontspec}
  \fi
  \defaultfontfeatures{Ligatures=TeX,Scale=MatchLowercase}
\fi
% use upquote if available, for straight quotes in verbatim environments
\IfFileExists{upquote.sty}{\usepackage{upquote}}{}
% use microtype if available
\IfFileExists{microtype.sty}{%
\usepackage{microtype}
\UseMicrotypeSet[protrusion]{basicmath} % disable protrusion for tt fonts
}{}
\usepackage{hyperref}
\hypersetup{unicode=true,
            pdftitle={CovSTATIS: The basis of multi-table techniques for group and individual connectivity analyses},
            pdfauthor={Jenny Rieck~\& Derek Beaton},
            pdfkeywords={keywords},
            pdfborder={0 0 0},
            breaklinks=true}
\urlstyle{same}  % don't use monospace font for urls
\usepackage{graphicx,grffile}
\makeatletter
\def\maxwidth{\ifdim\Gin@nat@width>\linewidth\linewidth\else\Gin@nat@width\fi}
\def\maxheight{\ifdim\Gin@nat@height>\textheight\textheight\else\Gin@nat@height\fi}
\makeatother
% Scale images if necessary, so that they will not overflow the page
% margins by default, and it is still possible to overwrite the defaults
% using explicit options in \includegraphics[width, height, ...]{}
\setkeys{Gin}{width=\maxwidth,height=\maxheight,keepaspectratio}
\IfFileExists{parskip.sty}{%
\usepackage{parskip}
}{% else
\setlength{\parindent}{0pt}
\setlength{\parskip}{6pt plus 2pt minus 1pt}
}
\setlength{\emergencystretch}{3em}  % prevent overfull lines
\providecommand{\tightlist}{%
  \setlength{\itemsep}{0pt}\setlength{\parskip}{0pt}}
\setcounter{secnumdepth}{0}
% Redefines (sub)paragraphs to behave more like sections
\ifx\paragraph\undefined\else
\let\oldparagraph\paragraph
\renewcommand{\paragraph}[1]{\oldparagraph{#1}\mbox{}}
\fi
\ifx\subparagraph\undefined\else
\let\oldsubparagraph\subparagraph
\renewcommand{\subparagraph}[1]{\oldsubparagraph{#1}\mbox{}}
\fi

%%% Use protect on footnotes to avoid problems with footnotes in titles
\let\rmarkdownfootnote\footnote%
\def\footnote{\protect\rmarkdownfootnote}


  \title{CovSTATIS: The basis of multi-table techniques for group and individual
connectivity analyses}
    \author{Jenny Rieck\textsuperscript{1}~\& Derek Beaton\textsuperscript{1}}
    \date{}
  
\shorttitle{CovSTATIS for connectivity}
\affiliation{
\vspace{0.5cm}
\textsuperscript{1} Rotman Research Institute, Baycrest Health Sciences}
\keywords{keywords\newline\indent Word count: X}
\usepackage{csquotes}
\usepackage{upgreek}
\captionsetup{font=singlespacing,justification=justified}

\usepackage{longtable}
\usepackage{lscape}
\usepackage{multirow}
\usepackage{tabularx}
\usepackage[flushleft]{threeparttable}
\usepackage{threeparttablex}

\newenvironment{lltable}{\begin{landscape}\begin{center}\begin{ThreePartTable}}{\end{ThreePartTable}\end{center}\end{landscape}}

\makeatletter
\newcommand\LastLTentrywidth{1em}
\newlength\longtablewidth
\setlength{\longtablewidth}{1in}
\newcommand{\getlongtablewidth}{\begingroup \ifcsname LT@\roman{LT@tables}\endcsname \global\longtablewidth=0pt \renewcommand{\LT@entry}[2]{\global\advance\longtablewidth by ##2\relax\gdef\LastLTentrywidth{##2}}\@nameuse{LT@\roman{LT@tables}} \fi \endgroup}


\DeclareDelayedFloatFlavor{ThreePartTable}{table}
\DeclareDelayedFloatFlavor{lltable}{table}
\DeclareDelayedFloatFlavor*{longtable}{table}
\makeatletter
\renewcommand{\efloat@iwrite}[1]{\immediate\expandafter\protected@write\csname efloat@post#1\endcsname{}}
\makeatother
\usepackage{lineno}

\linenumbers

\authornote{An author note

}

\abstract{
Some abstract


}

\usepackage{amsthm}
\newtheorem{theorem}{Theorem}[section]
\newtheorem{lemma}{Lemma}[section]
\theoremstyle{definition}
\newtheorem{definition}{Definition}[section]
\newtheorem{corollary}{Corollary}[section]
\newtheorem{proposition}{Proposition}[section]
\theoremstyle{definition}
\newtheorem{example}{Example}[section]
\theoremstyle{definition}
\newtheorem{exercise}{Exercise}[section]
\theoremstyle{remark}
\newtheorem*{remark}{Remark}
\newtheorem*{solution}{Solution}
\begin{document}
\maketitle

\hypertarget{introduction}{%
\section{Introduction}\label{introduction}}

\hypertarget{methods}{%
\section{Methods}\label{methods}}

\hypertarget{covstatis}{%
\subsection{CovSTATIS}\label{covstatis}}

Here we primarily describe CovSTATIS, followed briefly by an explanation
of CovSTATIS for distance matrices (DiSTATIS). We then discuss an
extension of both techniques called \enquote{K+1 CovSTATIS} and
\enquote{K+1 DiSTATIS}, respectively, where there is an external or
additional \enquote{target} table (the \enquote{\(K+1\)} amongst \(K\)
tables).

\hypertarget{notation}{%
\subsubsection{Notation}\label{notation}}

Our notation rougly matches that of (CITE), but introduces slight
differences and new information for our particular explanation. Bold
uppercase letters denotes matrices (e.g., \(\mathbf{R}\)), bold
lowercase letters denote vectors (e.g., \(\bf{r}\)), and italic
lowercase letters denote specific elements (e.g., \(r\)). Upper case
italic letters denote cardinality, size, or length (e.g., \(I\)). For
example, \({\bf R}\) is an \(I \times I\) matrix, which has \(I\) rows
and \(I\) columns. Lower case subscript italic denotes a specific index
(e.g., \(_i\)). A generic element of \(\mathbf{R}\) at the \(_{i}\)th
row and \(_{j}\)th column is \(x_{i,j}\). Common letters of varying type
faces, for example \({\bf R}\), \(\bf{r}\), \(r_{i,j}\), come from the
same data struture. Vectors are assumed to be column vectors unless
otherwise specified. Two matrices side-by-side denotes standard matrix
multiplication (e.g., \(\bf{R}\bf{Z}\)), where \(\odot\) denotes
element-wise (Hadamard) multiplication where \(\oslash\) denotes
element-wise (Hadamard) division. Superscript \(^{T}\) denotes the
transpose operation, superscript \(^{-1}\) denotes standard matrix
inversion, and superscript \(^{+}\) denotes the Moore-Penrose
pseudo-inverse. The diagonal operation, \(\mathrm{diag\{\}}\),
transforms a vector into a diagonal matrix, or extracts the diagonal of
a matrix and produces a vector. The vectorize operation,
\(\mathrm{vec\{\}}\), transforms a matrix into a column vector.

The following letters have reserved and specific meanings. \({\bf I}\)
denotes the identity matrix. \({\bf R}\) denotes a generic but
individual covariance (or correlation) matrix, where \({\bf R}_{[k]}\)
denotes the \(_k\)th covariance matrix in a set of \(K\) covariance
matrices. \({\bf S}\) denotes a generic but individual cross-product
matrix, where \({\bf S}_{[k]}\) denotes the \(_k\)th covariance matrix
in a set of \(K\) covariance matrices. \({\bf D}\) denotes a generic but
individual distance matrix, where \({\bf D}_{[k]}\) denotes the \(_k\)th
covariance matrix in a set of \(K\) covariance matrices. \({\bf L}\) is
an external (\enquote{K+1}) covariance matrix of exactly the same size
as any of the \({\bf R}_{[k]}\) matrices. We discuss the properties of
these tables when needed.

\hypertarget{covstatis-1}{%
\subsubsection{CovSTATIS}\label{covstatis-1}}

CovSTATIS is a particular form of cross-product STATIS that works for
covariance (or correlation or equivalent) matrices. By definition, all
individual connectivity matrices must meet the following conditions: (1)
square (same number of rows and columns), (2) symmetric (upper and lower
off-diagonal triangles are identical), and (3) that after a single
preprocessing step (double centering), each matrix is positive
semi-definite (strictly non-negative eigenvalues) just as in standard
PCA and related techniques (CITE).

For simplicity let us assume the use of correlation matrices (which are
covariance matrices of normalized column vectors). First, each
\({\bf R}_{[k]}\) must be converted to a cross-product matrix through
double centering. We denote a centering matrix as
\({\boldsymbol \Xi} = \mathbf{I} - \mathbf{1}(I^{-1})\mathbf{1}^{T}\).
We then double center each \({\bf R}_{[k]}\) to convert them to
cross-product matrices as

\begin{equation}
{\bf S}_{[k]} = \frac{1}{2}{\boldsymbol \Xi}{\bf R}_{[k]}{\boldsymbol \Xi}.
\end{equation}

Next, each \(I \times I\) \({\bf R}_{[k]}\) matrix is normalized to
account for different variance per each of the \(K\) tables. In general
there are three strategies: no normalization (if the assumed variance
per table is roughly equal), to normalize each table such that the sum
of all squared elements equals 1, or a dividing each table by its first
eigenvalue (i.e., \enquote{MFA normalization}). While there are other
normalization strategies, these are the most typically used; see (CITE,
CITE) for more details on normalization strategies per table in
multi-table PCA-like analyses. Thus we assume that each
\({\bf S}_{[k]}\) is a normalized cross product matrix. MFA
normalization is the preferred normalization approach.

We then must compute a similarity between all pairs of matrices in order
to obtain \enquote{\(\alpha\) weights} for each table. We first
vectorize all \({\bf S}_{[k]}\) matrices and store each column vector in
a new matrix as

\begin{equation}
\mathbf{Z} = [ \mathrm{vec\{ {\bf S}_{[1]}, \dots, {\bf S}_{[k]}, \dots, {\bf S}_{[K]}   \}} ]
\end{equation}

where \(\mathbf{Z}\) is of size \(I^{2} \times K\) and
\(\mathbf{C} = \mathbf{Z}^{T}\mathbf{Z}\). We can obtain the \(\alpha\)
weights in one of two ways: (1) through the eigenvalue decomposition
(EVD) of \(\mathbf{C}\) as
\(\mathbf{C} = \mathbf{V}\boldsymbol{\Theta}\mathbf{V}^{T}\) or
alternatively through the singular value decomposition (SVD) of
\({\mathbf Z}\):

\begin{equation}
\mathbf{Z} = \mathbf{U}\boldsymbol{\Delta}\mathbf{V}^{T}.
\end{equation} The \(\mathbf{V}\) in both the EVD or SVD approaches are
equivalent, and \(\boldsymbol{\Delta}^2 = \boldsymbol{\Theta}\). We
compute the \(\alpha\) weights from the first vector of \(\mathbf{V}\).
To note, all elements in the first vector of \(\mathbf{V}\) have the
same sign and reflect overall similarity; the large the value the more
similar that table is to all other tables. We compute the \(alpha\)
weights as

\begin{equation}
\boldsymbol{\alpha} = \mathbf{v}_{1}  \times (\mathbf{v}_{1}^{T}\mathbf{1})^{-1}
\end{equation}

Alternatively we can compute similarity between the normalized (i.e.,
Z-scored) column vectors of \(\mathbf{Z}\), which is equivalent to
computing the Rv coefficient (CITE) between each table. This
Rv-similarity is the preferred approach to compute the \(\alpha\)
weights.

We then compute the compromise cross-product matrix as

\begin{equation}
\mathbf{S}_{[+]} = \sum\limits_{i=1}^K \alpha_{k}\mathbf{S}_{[k]}.
\end{equation}

Finally we decompose \(\mathbf{S}_{[+]}\) with the EVD as follows

\begin{equation}
\mathbf{S}_{[+]} = \mathbf{Q}\boldsymbol{\Lambda}\mathbf{Q}^{T}.
\end{equation}

{[}Algorithm goes here{]} 1. Double center each

\begin{enumerate}
\def\labelenumi{\arabic{enumi}.}
\setcounter{enumi}{1}
\item
  Normalize
\item
  Check SSPSD
\item
  Vectorize to obtain Z
\item
  Get \(\alpha\) from svd(Z)
\item
  Compute S+ compromise
\item
  eigen(S+)
\item
  Compute compromise component scores
\item
  Compute partial component scores
\end{enumerate}

\hypertarget{results}{%
\section{Results}\label{results}}

\hypertarget{discussion}{%
\section{Discussion}\label{discussion}}

\newpage

\hypertarget{references}{%
\section{References}\label{references}}

\begingroup
\setlength{\parindent}{-0.5in}
\setlength{\leftskip}{0.5in}

\hypertarget{refs}{}

\endgroup


\end{document}
